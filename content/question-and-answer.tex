\documentclass[]{article}
\usepackage[T1]{fontenc}
\usepackage{lmodern}
\usepackage{amssymb,amsmath}
\usepackage{ifxetex,ifluatex}
\usepackage{fixltx2e} % provides \textsubscript
% use upquote if available, for straight quotes in verbatim environments
\IfFileExists{upquote.sty}{\usepackage{upquote}}{}
\ifnum 0\ifxetex 1\fi\ifluatex 1\fi=0 % if pdftex
  \usepackage[utf8]{inputenc}
\else % if luatex or xelatex
  \ifxetex
    \usepackage{mathspec}
    \usepackage{xltxtra,xunicode}
  \else
    \usepackage{fontspec}
  \fi
  \defaultfontfeatures{Mapping=tex-text,Scale=MatchLowercase}
  \newcommand{\euro}{€}
\fi
% use microtype if available
\IfFileExists{microtype.sty}{\usepackage{microtype}}{}
\ifxetex
  \usepackage[setpagesize=false, % page size defined by xetex
              unicode=false, % unicode breaks when used with xetex
              xetex]{hyperref}
\else
  \usepackage[unicode=true]{hyperref}
\fi
\hypersetup{breaklinks=true,
            bookmarks=true,
            pdfauthor={},
            pdftitle={},
            colorlinks=true,
            citecolor=blue,
            urlcolor=blue,
            linkcolor=magenta,
            pdfborder={0 0 0}}
\urlstyle{same}  % don't use monospace font for urls
\setlength{\parindent}{0pt}
\setlength{\parskip}{6pt plus 2pt minus 1pt}
\setlength{\emergencystretch}{3em}  % prevent overfull lines
\setcounter{secnumdepth}{0}
\usepackage{xepersian}

\settextfont[Scale=1]{XB Zar}%{XB Niloofar}%{B Nazanin}%
\setdigitfont{XB Zar}%{XB Niloofar}%{B Nazanin}%
\setlatintextfont[Scale=1]{XB Zar}%{Times New Roman}%
\defpersianfont\nastaliq[Scale=1.5]{IranNastaliq}

\title{طرح
پیشنهادی پرسش-و-پاسخ}


\author{}
\date{۱۳۹۴/۰۲/۲۰}


\begin{document}
\maketitle



\textbf{گوشزد ۱:} این نوشته برای ارائه به مرکز تحقیقات و نشر معارف
اهل‌البیت یا هر شخصیّت حقیقی/حقوقی جهت تأمین هزینه‌های طرح، تنظیم شده
است. 

\textbf{گوشزد ۲:} در این نوشته، طرح به صورت کلّی بیان شده و از بیان
جزئیات پرهیز شده است.

\

پیامبر خدا صلی الله علیه و اله و سلم فرمودند:
\begin{quote}
وَ قَالَ ص اَلْعِلْمُ خَزَائِنُ وَ مَفَاتِيحُهُ اَلسُّؤَالُ فَاسْأَلُوا
رَحِمَكُمُ اَللَّهُ فَإِنَّهُ تُؤْجَرُ أَرْبَعَةٌ اَلسَّائِلُ وَ
اَلْمُتَكَلِّمُ وَ اَلْمُسْتَمِعُ وَ اَلْمُحِبُّ لَهُمْ. /تحف العقول ص
۴۱
\end{quote}

\begin{quote}
دانش گنجينه هايي است كه كليدهايش پرسش است، پس بپرسيد-خداوند شما را رحمت
كناد ، زيرا در اين پرسش و پاسخ چهار كس اجر و مزد گيرند: پرسشگر، پاسخگو،
شنونده و دوستدار آنان.
\end{quote}

\subsection{هدف}\label{ux647ux62fux641}

\textbf{هدف}: طراحی سیستم پرسش-و-پاسخی که هر کسی بتواند آزادانه سؤال خود
را مطرح کند و دیگران بتوانند سؤال او را پاسخ دهند.

\subsection{مزایا}\label{ux645ux632ux627ux6ccux627}

\begin{enumerate}
\def\labelenumi{\arabic{enumi}.}
\itemsep1pt\parskip0pt\parsep0pt
\item
  \textbf{پرداختن به سخن نه به صاحب سخن} در فضای مباحثه در صورتی که هویت
  و کارنامهٔ علمی دو طرف گفت-و-گو از یکدیگر پوشیده باشد، باعث می‌شود که
  دو طرف گفت-و-گو با یکدیگر در مورد آنچه به بحث گذاشته می‌شود تمرکز
  کنند.
\item
  \textbf{اجازهٔ طرح و بیان مسئله به افرادی که قدرت بیان مناسبی ندارند
  داده می‌شود} با توجه به اینکه عده‌ای با قاطعیّت و صلابت نظرات خود را
  به راحتی در فضای مباحثهٔ حضوری بیان می‌کنند اما افرادی وجود دارند که
  برای بیان افکار خود قدرت بیان خوبی ندارند، فضای مجازی به این افراد کمک
  می‌کند تا خارج از جوّ هیاهوی مباحثه، افکار خود را بیان کنند.
\item
  \textbf{بایگانی شدن بحث‌ها برای استفادهٔ دیگران بدون شتابزدگی} این
  سیستم‌ها به کاربران این فرصت را می‌دهد که خارج از هیاهوی مباحثه، در
  فرصت مناسب مباحث مطرح شده بین افراد را مطالعه کند.
\item
  \textbf{میدان دادن به افرادی که از بازگو کردن سؤالات خود واهمه دارند}
  با توجه به اینکه بسیاری از مردم، از پرسیدن سؤالات بنیادین و به چالش
  کشیدن باورهای عمومی و رایج واهمه دارند، ایجاد فضایی که مردم بتوانند
  بدون ابراز هویّت خود، اندیشه‌ها و سؤالات خود را مطرح کنند کمک به باز
  شدن فضای گفتمان در جامعه خواهد شد.
\item
  \textbf{گذر از فضای جدلی و معطوف-به-قدرت به فضای هم‌اندیشی و
  معطوف-به-حقیقت} تفکر کلامی (تئولوژیک)، در پی اثبات باورهای خود است نه
  در پی کشف حقیقت. از این رو افراد پاسخگو در طرح‌های پاسخگویی به شبهات،
  در صدد پاسخ‌گویی و اثبات باورهای خود به مخاطبین هستند و فضای گفتمان و
  هم‌اندیشی با مخاطب خود برقرار نمی‌کنند همین موضوع باعث می‌شود که نه
  تنها حقیقت برای طرفین به خوبی منکشف نشود، بلکه باعث دفع مخاطب برای بحث
  و گفت-و-گو می‌شود. از این رو بهتر است که داعیهٔ پاسخگویی نداشته باشیم
  بلکه بستری مناسب برای گفتمان فراهم کنیم.
\end{enumerate}

\subsection{موضوعات
پرسش-و-پاسخ}\label{ux645ux648ux636ux648ux639ux627ux62a-ux67eux631ux633ux634-ux648-ux67eux627ux633ux62e}

هدف اصلی از طراحی چنین وب‌گاهی، ایجاد فضایی برای هم‌اندیشی و گفتمان در
زمینه‌های اعتقادی و هستی‌شناسی‌ست. اما با توجه به سابقهٔ سایت‌هایی مثل
استک‌اکسچنج و کورا، می‌توان به مرور در تمامی زمینه‌های علمی و مورد نیاز،
زیردامنه‌هایی برای بحث و گفت-و-گو در آن موضوع فراهم کرد. موارد زیر
نمونه‌ای از این موضوعات می‌تواند باشد:

\begin{itemize}
\itemsep1pt\parskip0pt\parsep0pt
\item
  قرآن
\item
  احادیث
\item
  سلامت
\item
  خانواده و تربیت فرزند
\item
  آشپزی
\item
  ورزش و تفریح
\item
  تلویزیون و فیلم
\item
  فلسفه
\item
  عرفان
\end{itemize}

\subsection{کارهای
فعلی}\label{ux6a9ux627ux631ux647ux627ux6cc-ux641ux639ux644ux6cc}

کارهای انجام شدهٔ فعلی اکثراً حالت هم‌اندیشی ندارند:

\begin{itemize}
\itemsep1pt\parskip0pt\parsep0pt
\item
  \href{http://porseman.org/}{پرسمان}
\item
  \href{http://www.pasokhgoo.ir/}{مرکز ملّی پاسخگویی به سؤالات}
\item
  \href{http://www.islamquest.net/}{اسلام کوئست}
\item
  و \ldots{}
\end{itemize}

اما نمونه‌های خوب که مدّنظر ماست:

\begin{itemize}
\itemsep1pt\parskip0pt\parsep0pt
\item
  \href{http://islam.stackexchange.com}{اسلام در استک‌اکسچنج} و همچنین
  بقیهٔ زیردامنه‌های استک‌اکسچنج
\item
  \href{http://www.quora.com/}{کورا}
\end{itemize}

که در میان این نمونه‌های خوب، نمونهٔ مناسبی برای زبان فارسی وجود ندارد،
به همین جهت مناسب است که در راه‌اندازی چنین سایتی تلاش کنیم.

و در انتها، این نوشته را با جملهٔ زیبایی از هیدگر به پایان می‌رسانم که
می‌گفت «پرسش، تقوی تفکر است».

\end{document}
