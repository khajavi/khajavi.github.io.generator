\documentclass[]{article}
\usepackage[T1]{fontenc}
\usepackage{lmodern}
\usepackage{amssymb,amsmath}
\usepackage{ifxetex,ifluatex}
\usepackage{fixltx2e} % provides \textsubscript
% use upquote if available, for straight quotes in verbatim environments
\IfFileExists{upquote.sty}{\usepackage{upquote}}{}
\ifnum 0\ifxetex 1\fi\ifluatex 1\fi=0 % if pdftex
  \usepackage[utf8]{inputenc}
\else % if luatex or xelatex
  \ifxetex
    \usepackage{mathspec}
    \usepackage{xltxtra,xunicode}
  \else
    \usepackage{fontspec}
  \fi
  \defaultfontfeatures{Mapping=tex-text,Scale=MatchLowercase}
  \newcommand{\euro}{€}
\fi
% use microtype if available
\IfFileExists{microtype.sty}{\usepackage{microtype}}{}
\ifxetex
  \usepackage[setpagesize=false, % page size defined by xetex
              unicode=false, % unicode breaks when used with xetex
              xetex]{hyperref}
\else
  \usepackage[unicode=true]{hyperref}
\fi
\hypersetup{breaklinks=true,
            bookmarks=true,
            pdfauthor={},
            pdftitle={},
            colorlinks=true,
            citecolor=blue,
            urlcolor=blue,
            linkcolor=magenta,
            pdfborder={0 0 0}}
\urlstyle{same}  % don't use monospace font for urls
\setlength{\parindent}{0pt}
\setlength{\parskip}{6pt plus 2pt minus 1pt}
\setlength{\emergencystretch}{3em}  % prevent overfull lines
\setcounter{secnumdepth}{0}

\usepackage{xepersian}

\title{طرح
پیشنهادی نرم‌افزار شرح واژگان قرآن}
%\author{میلاد خواجوی}
\date{۱۳۹۴/۰۲/۲۰}

\begin{document}
\maketitle

\textbf{گوشزد ۱:} این نوشته برای ارائه به مرکز تحقیقات و نشر معارف
اهل‌البیت یا هر شخصیّت حقیقی/حقوقی جهت تأمین هزینه‌های طرح، تنظیم شده
است.

\textbf{گوشزد ۲:} در این نوشته، طرح به صورت کلّی بیان شده و از بیان
جزئیات پرهیز شده است.

در حال حاضر برای خواندن متن قرآن نرم‌افزارهایی از جمله
\href{http://tanzil.net}{تنزیل}، \href{quran.com}{قرآن دات کام} و
\href{http://quran.ahlolbait.com/}{قرآن اهل‌البیت} موجود است. این
پروژه‌ها معنی تمام آیه را به زبا‌ن‌های مختلف قرار می‌دهند اما معنای
واژگان قرآن را ارائه نمی‌دهند. البته پروژه‌های

\begin{itemize}
\itemsep1pt\parskip0pt\parsep0pt
\item
  \href{http://alpha.quran.com/}{آلفا دات قرآن دات کام}
\item
  \href{http://www.almaany.com/quran}{المعانی}
\end{itemize}

معانی واژگان را فقط به زبان انگلیسی ارائه می‌دهند که این کار برای
فارسی‌زبان‌ها و دیگر کاربران غیر‌انگلیسی زبان کافی نیست.

پیشنهاد می‌شود که پروژه‌ای راه‌اندازی شود دارای ویژگی‌های زیر باشد.

\begin{enumerate}
\def\labelenumi{\arabic{enumi}.}
\itemsep1pt\parskip0pt\parsep0pt
\item
  برای جذب مشارکت عمومی حتماً به صورت نرم‌افزار آزاد ارائه شود.
\item
  در فاز اول برای کاربران فارسی‌زبان امکان نمایش ترجمهٔ تک تک واژگان
  قرآن از عربی به فارسی را داشته باشد. به طوری که کاربر با کلیک بر روی
  تک تک واژگان بتواند به صورت تعاملی برابر فارسی واژهٔ قرآنی را مشاهد
  کند. برای مثال مراجعه کنید به \href{http://alpha.quran.com/}{نمونهٔ
  انگلیسی آن}
\item
  در فازهای بعدی با جذب برنامه‌نویسان از سرتاسر دنیا نرم‌افزار را برای
  دیگر زبان‌ها آماده کرد.
\end{enumerate}

برای دریافت داده‌ها می‌توان با ناشران کتاب‌های
\href{http://bookroom.ir/book/5380}{قرآن با ترجمه و شرح واژگان آقای
بهرام‌پور} و
\href{http://makarem.ir/newmain.aspx?lid=0\&mid=61913\&CatID=6509\&typeinfo=3}{لغات
در تفسیر نمونه} همکاری کرد.

این پروژه متعلق به هیچ سازمانی نخواهد بود، اما افراد حقیقی/حقوقی مانند
خیرین، سازمان‌ها و شرکت‌ها می‌توانند در تهیهٔ هزینه‌های آن مشارکت داشته
باشند.

\end{document}
